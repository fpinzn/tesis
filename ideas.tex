Analisis del hardware necesario
Costos de implementacion
Posibles usos
Alcances - Claramente se necesita un cave! monimoni!

Prubas?
Pruebas de usabilidad con -usuarios-, grabarlas? 
Diferencias entre usuarios diestros y zurdos
Usos para personas incapacitadas?

La metafora de la cámara
Reducir el numero de interfaces

Fragmentacion del software utilizado en immersive systems.
Dificultades

Decisiones de diseño de la UX.

Encontrar referencia de como limitar las posibildades del usuario puede mejorar la interacción y la experiencia. Relacionado con la cantidad de caracteres usados en las anotaciones -> Principio de pareto->100 Design principles

Posibles referencias?

Usabilidad:

	The use of everyday things?
	Interaction journals?






Buscar documentacion de:

Minimapas como ayuda en la navegacion 3D.

BOWMAN, D. A., KRUIJFF, E., LAVIOLA, J. J., AND POUPYREV, I. 2004. 3D User Interfaces: Theory and Practice. Addison Wesley Longman Publishing Co., Inc., Redwood City, CA, USA.

Ideas sueltas:

	La navegacion debe estar escalada al tamaño del zoom utilizado. Ej: Si en zoom x1 se mueve 50mts en x0.1 se debe mover 5.
	Cuando se quieran revisar las anotaciones ir al mismo punto exacto desde el cual se tomo la foto
	Quitar cosas del medio cuando ocluyen otras? Revisar Sonnet.