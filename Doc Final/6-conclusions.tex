\chapter{Conclusions}
\section{Methods and Result Validation}

This part of the document is a whole chapter in the ABET format. For my project, this part of the guidelines is particularly out of focus. I decline to waste space and time in such thing, so this small section is its replacement.

The method used to validate the application was a standard use test by Frédéric, Sébastien and other memebers of the team at the Institut Image that evidenced the correct behavior of the application and the fullfillment of the Functional and Non-Functional requirements.

As the deliverable of the project is an application with a straight forward architecture, and standard usability tests would take more time than we can afford.
%7.1	Discusión
%Resumen del trabajo, discutiendo el desempeño y las limitaciones; problemas encontrados y cómo pudieron o podrían resolverse; lo que falta por hacer.
%Validación de los resultados de forma cuantitativa y cualitativa.
\section{Discussion}
This work features a 3D annotation system to be used in immersive environments. The application make use of an Android powered Tablet, an IR tracking system, and computers running windows XP. The position and orientation of the tablet is calculated by the IR tracking system and based on this information a render scene is streamed to the tablet via a WiFi connection. This is thought to work as a window metaphor, so in the tablet screen is showed the rendered scene as it was a blank frame and being consistent with the image projected in the immersive system.  This, enables the user to make use of its natural intuition to interact in the physical world to ease the learning curve of the application, and use it in a more natural way. The user can position its viewpoint in a his desired location and associate an annotation with this point of view. Other users might review the annotation and its associated viewpoint to enable an agile workflow with the associated model.

There were to important issues during the development. The first one is the lack of selecting an specific region in a screenshot to associate the annotation, it is a simple issue that might be solved with a little development time and more C++ expertise. The other important issue is the window/camera metaphor calibration. This calibration is a pecise process and even the slightest imperfection is evident to the user, again more time and expertise would solve the problem.

%7.2	Trabajo futuro
%Lista de sugerencias para trabajos futuros. Enfatizar aquellos resultados que merezcan consideración especial (v.gr., casos especiales que deban ser tenidos en cuenta, necesidad de proteger la propiedad intelectual, etc.). 
\section{Future Work}
The most important future work that might be developed is the improvement of the camera/window metaphor, as its correct implementation would imply a huge leap forward in virtual reality systems. This enhancement would be made by developing a tool that enables an easy way to calibrate the system to display correctly the metaphor.

Another important future work, would be the ability to follow specific user's annotations and track the progress in the annotations of the model in a similar fashion as the code progress is displayed in the github repositories.

Of course the modification of the application, so it would be capable of running in a standard PC would broad it's reach and potentialities.