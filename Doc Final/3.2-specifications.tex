%3.2	Especificaciones 
%Establecer requerimientos precisos para el proyecto. Clasificar requerimientos en funcionales (entada / salida) y no funcionales (cualitativos, metodología esperada en la construcción de una solución, desempeño esperado de una solución, …).
%Discutir la posibilidad de aceptar soluciones aproximadas y niveles de aceptación (v.gr., deseable, aceptable) y justificar cuándo una solución inexacta o incompleta podría desarrollarse.
\section{Specifications}

\subsection{Functional and Non Functional Requirements}

\begin{center}

\begin{tabular}{p{12.6cm}}
	\hline
	{\Large Functional Requirement 1 (FR1)}\\ \hline \hline
	The user must be able to create annotations associated to the part of interest in the model. \\ \hline
	\large{Input:}\\ \hline
	The part of interest, the content of the annotation, its priority and its author.\\ \hline \hline
	\large{Output:}\\ \hline
	The annotation is successfully linked to the model.\\ \hline
\end{tabular}
\vspace{10mm}

\begin{tabular}{p{12.6cm}}
	\hline
	{\Large Functional Requirement 2 (FR2)}\\ \hline \hline
	The user must be able to retrieve the annotations made in the model. \\ \hline
	\large{Input:}\\ \hline
	None.\\ \hline \hline
	\large{Output:}\\ \hline
	The user can see a list of the annotations associated with de model.\\
	\hline
\end{tabular}\\
\vspace{10mm}

\begin{tabular}{p{12.6cm}}
	\hline
	{\Large Functional Requirement 3 (FR3)}\\ \hline \hline
	The user must be able to delete annotations in the model. \\ \hline
	\large{Input:}\\ \hline
	The annotation the user wishes to delete.\\ \hline \hline
	\large{Output:}\\ \hline
	The annotation is deleted.\\
	\hline
\end{tabular}\\
\vspace{10mm}



\begin{tabular}{p{12.6cm}}
	\hline
	{\Large Functional Requirement 4 (FR4)}\\ \hline \hline
	The user must save the annotations made in the model to a file. \\ \hline
	\large{Input:}\\ \hline
	The annotations.\\ \hline \hline
	\large{Output:}\\ \hline
	The annotations are saved to an external XML file.\\
	\hline
\end{tabular}\\
\vspace{10mm}

\begin{tabular}{p{12.6cm}}
	\hline
	{\Large Functional Requirement 5 (FR5)}\\ \hline \hline
	The application must load automatically the annotations made in the model from a file. \\ \hline
	\large{Input:}\\ \hline
	The archive containing the annotations.\\ \hline \hline
	\large{Output:}\\ \hline
	The user is now capable of seeing the annotations previously made.\\
	\hline
\end{tabular}\\
\vspace{10mm}



\begin{tabular}{p{12.6cm}}
	\hline
	{\Large Non Functional Requirement 1 (NFR1)}\\ \hline \hline
	The application must work in the cave-like systems available at the facilities of the Institut Image. \\ \hline
	\large{Origin:}\\ \hline
	Context of the development of the project.\\ \hline \hline
	\large{Level of Impact:}\\ \hline
	Sets the bases of the architecture of the system.\\
	\hline
\end{tabular}\\
\vspace{10mm}

\begin{tabular}{p{12.6cm}}
	\hline
	{\Large Non Functional Requirement 2 (NFR2)}\\ \hline \hline
	The application should work as well in a standard PC running Windows XP. \\ \hline
	\large{Origin:}\\ \hline
	The need to use the system in other locations with different facilities.\\ \hline \hline
	\large{Level of Impact:}\\ \hline
	Creates another branch of development inside the project.\\
	\hline
\end{tabular}\\
\vspace{10mm}

\begin{tabular}{p{12.6cm}}
	\hline
	{\Large Non Functional Requirement 3 (NFR3)}\\ \hline \hline
	The part of the application running in the cave servers must be developed using C++ and OpenSceneGraph. \\ \hline
	\large{Origin:}\\ \hline
	These default technologies used for high demanding applications and the \emph{de-facto} standard inside the laboratories related to the project.\\ \hline \hline
	\large{Level of Impact:}\\ \hline
	Sets the technologies to be used in the development.\\
	\hline
\end{tabular}\\
\vspace{10mm}

\begin{tabular}{p{12.6cm}}
	\hline
	{\Large Non Functional Requirement 4 (NFR4)}\\ \hline \hline
	The part of the application running in the cave servers must run in Windows XP. \\ \hline
	\large{Origin:}\\ \hline
	The OS of the cave is Windows XP.\\ \hline \hline
	\large{Level of Impact:}\\ \hline
	Sets the environment of development. Given the use of OpenSceneGraph as the base layer the impact is not as deep as it might have been.\\
	\hline
\end{tabular}\\
\vspace{10mm}

\begin{tabular}{p{12.6cm}}
	\hline
	{\Large Non Functional Requirement 5 (NFR5)}\\ \hline \hline
	The application must save the annotations in XML format. \\ \hline
	\large{Origin:}\\ \hline
	Asked by Frédéric, taking in account the advantages of the highly standardized format.\\ \hline \hline
	\large{Level of Impact:}\\ \hline
	Creates another module in the software to save and load annotations from an XML file.\\
	\hline
\end{tabular}\\
\vspace{10mm}


\begin{tabular}{p{12.6cm}}
	\hline
	{\Large Non Functional Requirement 6 (NFR6)}\\ \hline \hline
	The application will make use of the VRPN technology to interchange information between devices. \\ \hline
	\large{Origin:}\\ \hline
	Enhances the modifiability of the system and is a standard practice in both laboratories.\\ \hline \hline
	\large{Level of Impact:}\\ \hline
	Sets architectural constraints.\\
	\hline
\end{tabular}\\
\vspace{10mm}


\end{center}

Initially the parts of interest were exclusively polygons, I thought during the last stage of the development to broad the definition and include any point of view with a marquee highlighting the important sector, and drop the polygon selection due to it's hard of use and impact in the application workflow. The creation and associations of the marquees wasn't developed, so the application is capable of commenting points of view as the work presented by Duval \cite{Duval}.

\subsection{Acceptance Level}

The minimal accepted solution would be the one who fulfills all the requirements (functional and nonfunctional) mentioned before, except the Non Functional Requirement 2. This is, a solution that runs and is fully functional in cave like systems but doesn't work in a standard PC running windows XP.  

The optimal solution in the available time is then, the one described below on the requirements. The description of better global solutions will be reserved for the \emph{Future Work} part of this document as they involve a significantly major development work, timeline functionalities upon the models and their textual information, that are beyond the reach of this project.

An incomplete solution would be developed in the case that the calendar designed for developing the solution falls short due to unforeseen serious technical difficulties during the implementation. In this scenario the part of the project that would suffer the most, is without doubt the related to the reviewing of the annotations. This, because the first part that will be implemented is the one concerned with creating the annotations given it's need to afterwards see the annotations made.\footnote{Although the application was not fully developed, it's still an usable and reliable solution.}

On the other hand, another part of the project that could present non-optimal quality is the one concerned with the user experience. The project is thought to be a clean, minimal but satisfying solution to the problem it aims to solve, nevertheless a huge effort is being invested in improving the user experience and the quality attributes related to it. 

Taking in account the ideas presented in the famous book by the creators of the successful company 37signals \cite{37signals} and the Apple Development Guidelines \cite{apple}, a nice user experience involves reducing the span of functional requirements to the minimum necessary to successfully solve the problem, while focusing in less evident goods of the software. In the particular case of the project, the rendering power, the net throughput and the responsivity of the front-end (the Android powered tablet) are the most relevant factors in providing a satisfactory user experience. Is yet to be seen how easy or hard is to tune them right.